%%%%%%%%%%%%%%%%%%%%%%%%%%%%%%%%%%%%%%%%%%%%%%%%%%%%%%%%%%%%%%%%%%
%%%%%%%% ICML 2014 EXAMPLE LATEX SUBMISSION FILE %%%%%%%%%%%%%%%%%
%%%%%%%%%%%%%%%%%%%%%%%%%%%%%%%%%%%%%%%%%%%%%%%%%%%%%%%%%%%%%%%%%%

% Use the following line _only_ if you're still using LaTeX 2.09.
%\documentstyle[icml2014,epsf,natbib]{article}
% If you rely on Latex2e packages, like most moden people use this:
\documentclass{article}

% use Times
\usepackage{times}
% For figures
\usepackage{graphicx} % more modern
%\usepackage{epsfig} % less modern
\usepackage{subfigure} 

% For citations
\usepackage{natbib}

% For algorithms
\usepackage{algorithm}
\usepackage{algorithmic}

% As of 2011, we use the hyperref package to produce hyperlinks in the
% resulting PDF.  If this breaks your system, please commend out the
% following usepackage line and replace \usepackage{icml2014} with
% \usepackage[nohyperref]{icml2014} above.
\usepackage{hyperref}

% Packages hyperref and algorithmic misbehave sometimes.  We can fix
% this with the following command.
\newcommand{\theHalgorithm}{\arabic{algorithm}}

% Employ the following version of the ``usepackage'' statement for
% submitting the draft version of the paper for review.  This will set
% the note in the first column to ``Under review.  Do not distribute.''
\usepackage{icml2014} 


% The \icmltitle you define below is probably too long as a header.
% Therefore, a short form for the running title is supplied here:
\icmltitlerunning{TyckoPennLopezdelgado}

\begin{document} 

\twocolumn[
\icmltitle{Status Report: Predicting progression of Alzheimer's Disease with clinical and genotype data}

% It is OKAY to include author information, even for blind
% submissions: the style file will automatically remove it for you
% unless you've provided the [accepted] option to the icml2014
% package.
\icmlauthor{Josh Tycko}{joshtycko@gmail.com}
\icmlauthor{Spencer Penn}{spenn321@gmail.com}
\icmlauthor{Juan Jose Lopez Delgado}{juanlop@seas.upenn.edu}

% You may provide any keywords that you 
% find helpful for describing your paper; these are used to populate 
% the "keywords" metadata in the PDF but will not be shown in the document
\icmlkeywords{Alzheimer's disease, machine learning, ensemble methods, genotype}

\vskip 0.3in
]

\begin{abstract} 
Machine learning algorithms have the potential to predict Alzheimer's disease (AD) progression by analyzing large clinical and genomic datasets. Here, we describe our progress on our implementation of ensemble methods to generate accurate predictions from a large AD database. We are working with the data from 767 patients (split into training and testing sets), with plans to supplement our data with work from additional longitudinal studies. We have plans to gain domain expertise from the Penn doctors who originally developed the database we have accessed.

%FILL IN ABSTRACT HERE

\end{abstract} 



\section{Introduction}
Alzheimer'��s disease (AD) is predicted to affect 1 in 85 people globally by 2050, causing dementia and eventual death. Care in the US costs \$100 billion annually, and the available drugs can only help relieve some symptoms \cite{duthey13}.

\subsection{Motivation}
It is currently difficult to predict the progression of AD, and it often progresses undiagnosed for years. Machine learning algorithms have the potential to assist doctors and patients by accurately predicting disease progression based on clinical and genetic data, which would enable accurate, early diagnoses.

\subsection{Related work} Since the causes of AD are currently unknown and there are no laboratory tests that can accurately perform a diagnosis, AD progression is quantified with psychological tests like the mini-mental state examination (MMSE) - a questionnaire used to measure cognitive impairment. This set of 30 questions was developed in 1975 and remains the standard \cite{carolan07}

Machine learning algorithms have been used on ADNI data with varying success to predict the change in MMSE. Interestingly, no single algorithm has been shown to be superior across all AD datasets, particularly when progression is measured up to varying time points \cite{umer11}

\section{Materials \& Methods}
\subsection{Data} Data used in the preparation of this article were obtained from the Alzheimer's Disease Neuroimaging Initiative (ADNI) database (adni.loni.usc.edu). The data was collected over 2 years in 767 patients, including mental examinations and genotype in order to predict the progression of AD over time. Progression is quantified by the change in MMSE score over a 24 month period $(\Delta$MMSE$)$.

\subsection{Approach} We aim to develop an algorithm that is robustly accurate across data sets, by creating an ensemble model of the top models tried previously (simple logistic regression, random forests, and Bayesian nets). By weighting our ensemble with boosting, we will try to create an ensemble model that is superior in accuracy to any of the constituent models.

\section{Results}
% this is what we learned about the data
% this is our result from trying an algorithm
%include enough detail that someone could re-produce the result
%create one key figure

\section{Discussion}

\section{Next Steps}
\subsection{Integrating other learning algorithms}
%Fill in stuff

\subsection{Gaining Domain Expertise}
Several of the researchers who developed the ADNI database are here at Penn. We are being advised by Dr. Leslie Shaw and Dr. John Trojanowski on the best usage of the database and the relationship between the data features and the disease.


\section{Final Report}

Your final project report can be at most 5 pages long (include all text, appendices, figures, references, and anything else), and must be written in the provided \LaTeX\ template. 

At a minimum your final report must describe the problem/application and motivation, survey related work, discuss your approach, and describe your results/conclusions/impact of your project.  It should include enough detail such that someone else can reproduce your approach and results.  For inspiration on what should be included, see the project reports available on the links provided in Section~\ref{sect:Ideas}.  You will likely end up with a better report if you start by writing a 6-7 page report and then edit it down to 5 pages of well-written and concise prose.

In addition, your report must also include a figure that graphically depicts a major component of your project (e.g., your approach and how it relates to the application, etc.).  Such a summary figure makes your paper much more accessible by providing a visual counterpart to the text.  Developing such a concise and clear figure can actually be quite time-consuming; I often go through around ten versions before I end up with a good final version.

\section{Optional Suggestions for Your Paper and Formatting Guidance} 

\subsection{Figures}
 
You may want to include figures in the paper to help readers visualize
your approach and your results. Such artwork should be centered,
legible, and separated from the text. Lines should be dark and at
least 0.5~points thick for purposes of reproduction, and text should
not appear on a gray background.

Label all distinct components of each figure. If the figure takes the
form of a graph, then give a name for each axis and include a legend
that briefly describes each curve. Do not include a title inside the
figure; instead, be sure to include a caption describing your figure.

You may float figures to the top or
bottom of a column, and you may set wide figures across both columns
(use the environment {\tt figure*} in \LaTeX), but always place
two-column figures at the top or bottom of the page.

\subsection{Algorithms}

If you are using \LaTeX, please use the ``algorithm'' and ``algorithmic'' 
environments to format pseudocode. These require 
the corresponding stylefiles, algorithm.sty and 
algorithmic.sty, which are supplied with this package. 
Algorithm~\ref{alg:example} shows an example. 

\begin{algorithm}[tb]
   \caption{Bubble Sort}
   \label{alg:example}
\begin{algorithmic}
   \STATE {\bfseries Input:} data $x_i$, size $m$
   \REPEAT
   \STATE Initialize $noChange = true$.
   \FOR{$i=1$ {\bfseries to} $m-1$}
   \IF{$x_i > x_{i+1}$} 
   \STATE Swap $x_i$ and $x_{i+1}$
   \STATE $noChange = false$
   \ENDIF
   \ENDFOR
   \UNTIL{$noChange$ is $true$}
\end{algorithmic}
\end{algorithm}
 
\subsection{Tables} 
 
You may also want to include tables that summarize material. Like 
figures, these should be centered, legible, and numbered consecutively. 
However, place the title {\it above\/} the table, as in 
Table~\ref{sample-table}.
% Note use of \abovespace and \belowspace to get reasonable spacing 
% above and below tabular lines. 

\begin{table}[t]
\caption{Classification accuracies for naive Bayes and flexible 
Bayes on various data sets.}
\label{sample-table}
\vskip 0.15in
\begin{center}
\begin{small}
\begin{sc}
\begin{tabular}{lcccr}
\hline
\abovespace\belowspace
Data set & Naive & Flexible & Better? \\
\hline
\abovespace
Breast    & 95.9$\pm$ 0.2& 96.7$\pm$ 0.2& $\surd$ \\
Cleveland & 83.3$\pm$ 0.6& 80.0$\pm$ 0.6& $\times$\\
Glass2    & 61.9$\pm$ 1.4& 83.8$\pm$ 0.7& $\surd$ \\
Credit    & 74.8$\pm$ 0.5& 78.3$\pm$ 0.6&         \\
Horse     & 73.3$\pm$ 0.9& 69.7$\pm$ 1.0& $\times$\\
Meta      & 67.1$\pm$ 0.6& 76.5$\pm$ 0.5& $\surd$ \\
Pima      & 75.1$\pm$ 0.6& 73.9$\pm$ 0.5&         \\
\belowspace
Vehicle   & 44.9$\pm$ 0.6& 61.5$\pm$ 0.4& $\surd$ \\
\hline
\end{tabular}
\end{sc}
\end{small}
\end{center}
\vskip -0.1in
\end{table}

Tables contain textual material that can be typeset, as contrasted 
with figures, which contain graphical material that must be drawn. 
Specify the contents of each row and column in the table's topmost
row. Again, you may float tables to a column's top or bottom, and set
wide tables across both columns, but place two-column tables at the
top or bottom of the page.
 
%\subsection{Citations and References} 
%
%Please use APA reference format regardless of your formatter
%or word processor. If you rely on the \LaTeX\/ bibliographic 
%facility, use {\tt natbib.sty} and {\tt icml2014.bst} 
%included in the style-file package to obtain this format.
%
%Citations within the text should include the authors' last names and
%year. If the authors' names are included in the sentence, place only
%the year in parentheses, for example when referencing Arthur 
%'s
%pioneering work %\yrcite{Samuel59}. Otherwise place the entire
%reference in parentheses with the authors and year separated by a
%comma %\cite{Samuel59}. List multiple references separated by
%semicolons \cite{kearns89,Samuel59,mitchell80}. Use the `et~al.'
%construct only for citations with three or more authors or after
%listing all authors to a publication in an earlier reference \cite{MachineLearningI}.
%
%The references at the end of this document give examples for journal
%articles \cite{Samuel59}, conference publications \cite{langley00}, book chapters \cite{Newell81}, books \cite{DudaHart2nd}, edited volumes \cite{MachineLearningI}, 
%technical reports \cite{mitchell80}, and dissertations \cite{kearns89}. 
%
%Alphabetize references by the surnames of the first authors, with
%single author entries preceding multiple author entries. Order
%references for the same authors by year of publication, with the
%earliest first. Make sure that each reference includes all relevant
%information (e.g., page numbers).

 
\section*{Acknowledgments} 
 
Data collection and sharing for this project was funded by the Alzheimer's Disease Neuroimaging Initiative (ADNI) (National Institutes of Health Grant U01 AG024904) and DOD ADNI (Department of Defense award number W81XWH-12-2-0012). ADNI data are disseminated by the Laboratory for Neuro Imaging at the University of Southern California.

\bibliography{status}
\bibliographystyle{icml2014}

\end{document} 


% This document was modified from the file originally made available by
% Pat Langley and Andrea Danyluk for ICML-2K. This version was
% created by Lise Getoor and Tobias Scheffer, it was slightly modified  
% from the 2010 version by Thorsten Joachims & Johannes Fuernkranz, 
% slightly modified from the 2009 version by Kiri Wagstaff and 
% Sam Roweis's 2008 version, which is slightly modified from 
% Prasad Tadepalli's 2007 version which is a lightly 
% changed version of the previous year's version by Andrew Moore, 
% which was in turn edited from those of Kristian Kersting and 
% Codrina Lauth. Alex Smola contributed to the algorithmic style files.  
